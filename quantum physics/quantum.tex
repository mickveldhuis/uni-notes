\documentclass[a4paper]{article}

\usepackage[english]{babel}
\usepackage[utf8]{inputenc}
\usepackage{amsmath}
\usepackage{amssymb}
\usepackage{caption}
\usepackage{siunitx}
\usepackage{graphicx}
\usepackage{textcomp}
\usepackage{gensymb}
\graphicspath{{./}}
\usepackage[colorinlistoftodos]{todonotes}
\usepackage[section]{placeins}
\setlength{\parindent}{0pt}
\usepackage{url}
\usepackage{bm}
\usepackage{mathtools}
\usepackage{enumerate}

\title{Lecture Notes Quantum Physics}
\author{Mick Veldhuis}
\date{\today}

\begin{document}
\maketitle

\tableofcontents

\section{The wave function}

In QM we want to find the \textbf{wave function} $\Psi(x, t)$, a function of position and time. We can obtain this function by solving the \textbf{Schrodinger equation}:

\begin{equation}
    i\hbar\frac{\partial\Psi}{\partial t}=-\frac{\hbar^2}{2m}\frac{\partial^2}{\partial x^2} + V(x)\Psi
\end{equation}

where $\hbar=h/2\pi$. Additionally we also have the complex conjugate of the above equation, which is given as

\begin{equation}
    \frac{\partial\Psi^{*}}{\partial t}=-\frac{i\hbar}{2m}\frac{\partial^2}{\partial x^2} + \frac{i}{\hbar}V(x)\Psi
\end{equation}

\subsection{Statistical interpretation of $\Psi$}

Born's interpretation tells us that $|\Psi|^2=\Psi \Psi^{*}$ gives the probability of finding the particle at a some point $x$, at some time $t$. Or rather the probability of finding the particle (whose wave function we are considering) between $a$ and $b$, at some time $t$ is given by

\begin{equation}
    \int_{a}^{b}|\Psi(x,t)|^2\,dx
\end{equation}

From this interpretation also follows that QM is nondeterministic, as QM only offers statistical information about the possible results. One could then ask if one measures the position of the particle somewhere, where was it before? The commonly accepted view (that is the \textbf{Copenhagen interpretation}) says that the particle was not really anywhere. And thus the result of probing caused the specific result, probing the particle again will yield the same position. We say that the wave function collapses upon measurement.

\subsection{Some probability definitions}

\subsubsection{Discrete variables}

In general, the average value of some function $f$ is given by

\begin{equation}
    \langle f(j) \rangle = \sum_{j=0}^{\infty}f(j)P(j)
\end{equation}

The variance is given by

\begin{equation}
    \sigma^2 = \langle j^2 \rangle - \langle j \rangle^2
\end{equation}

so $\langle j^2 \rangle \ge \langle j \rangle^2$.

\subsubsection{Continuous variables}

Given that $\rho(x)\,dx$ is the probability that an individual lies between $x$ and $x+dx$, where $\rho(x)$ is called the probability density. So the probability that $x$ lies between $a$ and $b$ is given by

\begin{equation}
    P_{ab} = \int_{a}^{b}\rho(x)\,dx
\end{equation}

Such that

\begin{equation}
    \langle f(x) \rangle = \int_{-\infty}^{\infty}f(x)\rho(x)\,dx
\end{equation}

and 

\begin{equation}
    \sigma^2 \equiv \langle x^2 \rangle - \langle x \rangle^2
\end{equation}

\subsection{Normalization}

Since the total probability of anything is one, a logical requirement is that

\begin{equation}
    \int_{-\infty}^{\infty}|\Psi(x,t)|^2\,dx=1
\end{equation}

Such that if that integral does not yield one, but some other finite answer, we shall scale it with some real number $A$, like $A\Psi(x,t)$. If the result is not finite, the function is non normalizable and as such not physical. Note that the wave function should go to zero at both infinities. And $\Psi(x,t)$ should fall of faster than $1/\sqrt{|x|}$ in order to be normalizable. It is also important to note that if a wave function is normalized at any time $t$ it is also normalized for all future time.

\subsection{Momentum}

The expectation value of $x$ is given by

\begin{equation}
    \langle x \rangle = \int x\Psi^{*}\Psi\,dx
\end{equation}

Such that that of the speed $v$ is given as

\begin{equation}
    \langle v \rangle = \frac{d\langle x\rangle}{dt}=\int x\frac{\partial}{\partial t}\big(\Psi^{*}\Psi\big)\,dx=\frac{-i\hbar}{m}\int\Psi^{*}\frac{\partial\Psi}{\partial x}\,dx
\end{equation}

Such that the expectation value of momentum is given as

\begin{equation}
    \langle p \rangle = m\langle v \rangle = -i\hbar\int\Psi^{*}\frac{\partial\Psi}{\partial x}\,dx
\end{equation}

From these equations we extrapolate that there are two important operators, namely $x$, representing position, and $-i\hbar(\partial/\partial x)$ for momentum. In general any observable is a function of $x$ and $p$, $Q=Q(x, p)$. And in QM

\begin{equation}
    \langle Q \rangle = \int\Psi^{*}Q\bigg(x, -i\hbar\frac{\partial}{\partial x}\bigg)\Psi\,dx
\end{equation}

\subsection{H.U.P.}

A general property of waves is that they cannot have a definite position and a clear wavelength. In QM the wavelength of $\Psi$ is related to $p$ by the \textbf{de Broglie formula}:

\begin{equation}
    p=\frac{h}{\lambda}=\frac{2\pi h}{\lambda}
\end{equation}

So like waves, in QM we cannot know both the position and momentum in detail. It is one or the other. Quantitatively this is described by \textbf{Heisenberg's uncertainty principle}:

\begin{equation}
    \sigma_x\sigma_p\ge\frac{\hbar}{2}
\end{equation}

\section{Time-independent Schrodinger equation}

\subsection{Stationary states}

lets assume that $V=V(x)$, and let $\Psi(x,t)=\psi(x)\varphi(t)$. Now we can attempt to solve the Schrodinger equation by separation of variables.

\bigskip

\paragraph{1.} Note that

\begin{equation}
	\frac{\partial\Psi}{\partial t}=\psi\frac{d\varphi}{dt}\quad\text{and}\quad\frac{\partial^2\Psi}{\partial x^2}=\varphi\frac{d^2\psi}{dx^2}
\end{equation}

\paragraph{2.} Now we can write the Schrodinger equation as

\begin{equation}
	i\hbar\psi\frac{d\varphi}{dt}=-\frac{\hbar^2}{2m}\frac{d^2\psi}{dx^2}\varphi+V\psi\varphi
\end{equation}

\paragraph{3.} Then divide by $\psi\varphi$, and we obtain

\begin{equation}
i\hbar\frac{1}{\varphi}\frac{d\varphi}{dt}=-\frac{\hbar^2}{2m}\frac{d^2\psi}{dx^2}\frac{1}{\psi}+V
\end{equation}

Now the LHS is just a function of time and the RHS just a function of position.

\paragraph{4.} Now we let the separation constant be $E$, such that we obtain

\setlength{\leftskip}{0.5cm}

\paragraph{4.1}

\begin{equation}
	i\hbar\frac{1}{\varphi}\frac{d\varphi}{dt}=E\quad\Rightarrow\quad\varphi(t)=e^{-iEt/\hbar}
\end{equation}

\paragraph{4.2}

\begin{equation}
	-\frac{\hbar^2}{2m}\frac{d^2\psi}{dx^2}+V\psi=E\psi
\end{equation}

\setlength{\leftskip}{0cm}

This (last) equation is the time-independent Schrodinger equation; to solve it we first need to know $V(x)$.

\bigskip

One might ask why these separable solutions are so great, well

\begin{enumerate}
	\item They are stationary states: $|\Psi|^2$ and $\langle Q(x, p)\rangle$ do not depend on time, thus we can often drop $\varphi(t)$.
	\item They are states of definite total energy: In CM the Hamiltonian is defined as $$H=T+V=\frac{p^2}{2m}+V(x)$$ By applying the operator for the momentum as defined in section 1.4 we define the Hamiltonian operator: $$\hat{H}=-\frac{\hbar^2}{2m}\frac{\partial^2}{\partial x^2}+V$$ Such that the time-independent Schrodinger equation can be rewritten as $$\hat{H}\psi=E\psi$$ The expectation value $\langle H\rangle=E$, and note that $$\hat{H}^2\psi=E^2\psi$$ Such that $$\sigma_H^2=\langle H^2\rangle-\langle H\rangle^2=E^2-E^2=0$$ So a separable solution is certain to have a total energy $E$ for every measurement.
	\item The general solution is a linear combination of separable solutions: the time-independent Schrodinger equation yields an infinite collection of solutions $\{\psi_n(x)\}$, and each is associated with a specific energy $\{E_n\}$. Such that $$\Psi(x,t)=\sum_{n=1}^{\infty}c_n\psi_n(x)e^{iE_nt/\hbar}$$ Note that $|c_n|^2$ is the probability that a measurement of the energy returns $E_n$. And logically $$\sum |c_n|^2=1\quad\Rightarrow\quad\langle H\rangle=\sum |c_n|^2E_n$$
\end{enumerate}

\subsection{Infinite square well}

Lets now assume a certain potential

\[V(x) = \left\{
\begin{array}{lr}
0, & \text{if}\ 0 \le x \le a\\
\infty, & \text{otherwise}
\end{array}
\right.
\]

Thus the particle(s) is/are stuck between $x=0$ and $x=a$. 

\paragraph{Outside the well: } $\psi(x)=0$
\paragraph{Inside the well: }

\begin{align}
	V=0\quad&\Rightarrow\quad -\frac{\hbar^2}{2m}\frac{d^2\psi}{dx^2}=E\psi \\[1em]
	&\Rightarrow\quad \frac{d^2\psi}{dx^2}=-k^2\psi\ ,\ \ k^2=\frac{2mE}{\hbar^2} 
\end{align}

Note that we assume that $E\ge 0$. This is just a simple harmonic oscillator, thus the solution is given by

\begin{equation}
	\psi(x)=A\sin kx+B\cos kx
\end{equation} 

We have the boundary condition $\psi(0)=\psi(a)=0$, therefore $B=0$. Thus 

\begin{equation}
	\psi(x)=A\sin kx
\end{equation}

and since $A\not=0$ $ka=n\pi$. So

\begin{equation}
	k_n=\frac{n\pi}{a}
\end{equation}

Now the energies $E_n$ are given as

\begin{align}
	E_n=\frac{\hbar^2k_n^2}{2m}=\frac{\hbar^2\pi^2n^2}{2ma^2}
\end{align}

We arrive at this expression by noting that $E=T+V=T+0=p^2/2m$, and $p=2\pi\hbar/\lambda=k\hbar$ where $k$ is the wave number $k=2\pi/\lambda$. Such that $p_n=\hbar k_n\Rightarrow p_n^2=\hbar^2k_n^2$, which we then plug into $E_n=p_n^2/2m$.

\bigskip

Now we still need to find $A$, which we do by normalizing $\psi(x)$:

\begin{equation}
	|A|^2\int_0^a\sin^2 kx\,dx=|A|^2\frac{a}{2}=1\quad\Rightarrow\quad A=\sqrt{\frac{2}{a}}
\end{equation}

As such we now have $\psi_n(x)$

\begin{equation}
	\psi_n(x)=\sqrt{\frac{2}{a}}\sin\bigg(\frac{n\pi}{a}x\bigg)
\end{equation}

Where $\psi_1$ is the ground state and $\{\psi_n\}$, for $n>1$, are the excited states. A few properties of $\psi_n(x)$ are

\begin{enumerate}
	\item $\psi_n$ is alternately even and odd.
	\item As energy increases, each successive state has one more node (zero-crossing).
	\item $$\int\psi_m(x)^*\psi_n(x)\,dx=\delta_{mn}$$ Thus 0 for $m\not=n$ and 1 for $m=n$.
	\item They are complete, meaning that you can express any $f(x)$ as a linear combination of $\psi_n$, $$f(x)=\sum_{n=1}^{\infty}c_n\psi_n(x)\ , \ \ c_n=\int\psi_n(x)^*f(x)\,dx$$
\end{enumerate}

Thus, given the potential $V(x)$ we find $\Psi_n(x, t)$ to be

\begin{equation}
	\Psi_n(x,t)=\sqrt{\frac{2}{a}}\sin\bigg(\frac{n\pi}{a}x\bigg)e^{-iE_nt/\hbar}\ , \ \ E_n=\frac{\hbar^2\pi^2n^2}{2ma^2}
\end{equation}

And therefore the wave function $\Psi(x, t)$ is given by

\begin{equation}
\Psi(x,t)=\sum_{n=1}^{\infty}c_n\Psi_n(x,t)=\sum_{n=1}^{\infty}c_n\sqrt{\frac{2}{a}}\sin\bigg(\frac{n\pi}{a}x\bigg)e^{-iE_nt/\hbar}
\end{equation}

where

\begin{equation}
	c_n=\sqrt{\frac{2}{a}}\int_0^a\sin\bigg(\frac{n\pi}{a}x\bigg)\Psi(x,0)\,dx
\end{equation}

\end{document}

