\documentclass[a4paper]{article}

\usepackage[english]{babel}
\usepackage[utf8]{inputenc}
\usepackage{amsmath}
\usepackage{amssymb}
\usepackage{caption}
\usepackage{siunitx}
\usepackage{graphicx}
\usepackage{textcomp}
\usepackage{gensymb}
\graphicspath{{./}}
\usepackage[colorinlistoftodos]{todonotes}
\usepackage[section]{placeins}
\setlength{\parindent}{0pt}
\usepackage{url}
\usepackage{bm}
\usepackage{mathtools}
\usepackage{enumerate}

\title{Lecture Notes Quantum Physics}
\author{Mick Veldhuis}
\date{\today}

\begin{document}
\maketitle

\tableofcontents

\section{The wave function}

In QM we want to find the \textbf{wave function} $\Psi(x, t)$, a function of position and time. We can obtain this function by solving the \textbf{Schrodinger equation}:

\begin{equation}
    i\hbar\frac{\partial\Psi}{\partial t}=-\frac{\hbar^2}{2m}\frac{\partial^2}{\partial x^2} + V(x)\Psi
\end{equation}

where $\hbar=h/2\pi$. Additionally we also have the complex conjugate of the above equation, which is given as

\begin{equation}
    \frac{\partial\Psi^{*}}{\partial t}=-\frac{i\hbar}{2m}\frac{\partial^2}{\partial x^2} + \frac{i}{\hbar}V(x)\Psi
\end{equation}

\subsection{Statistical interpretation of $\Psi$}

Born's interpretation tells us that $|\Psi|^2=\Psi \Psi^{*}$ gives the probability of finding the particle at a some point $x$, at some time $t$. Or rather the probability of finding the particle (whose wave function we are considering) between $a$ and $b$, at some time $t$ is given by

\begin{equation}
    \int_{a}^{b}|\Psi(x,t)|^2\,dx
\end{equation}

From this interpretation also follows that QM is nondeterministic, as QM only offers statistical information about the possible results. One could then ask if one measures the position of the particle somewhere, where was it before? The commonly accepted view (that is the \textbf{Copenhagen interpretation}) says that the particle was not really anywhere. And thus the result of probing caused the specific result, probing the particle again will yield the same position. We say that the wave function collapses upon measurement.

\subsection{Some probability definitions}

\subsubsection{Discrete variables}

In general, the average value of some function $f$ is given by

\begin{equation}
    \langle f(j) \rangle = \sum_{j=0}^{\infty}f(j)P(j)
\end{equation}

The variance is given by

\begin{equation}
    \sigma^2 = \langle j^2 \rangle - \langle j \rangle^2
\end{equation}

so $\langle j^2 \rangle \ge \langle j \rangle^2$.

\subsubsection{Continuous variables}

Given that $\rho(x)\,dx$ is the probability that an individual lies between $x$ and $x+dx$, where $\rho(x)$ is called the probability density. So the probability that $x$ lies between $a$ and $b$ is given by

\begin{equation}
    P_{ab} = \int_{a}^{b}\rho(x)\,dx
\end{equation}

Such that

\begin{equation}
    \langle f(x) \rangle = \int_{-\infty}^{\infty}f(x)\rho(x)\,dx
\end{equation}

and 

\begin{equation}
    \sigma^2 \equiv \langle x^2 \rangle - \langle x \rangle^2
\end{equation}

\subsection{Normalization}

Since the total probability of anything is one, a logical requirement is that

\begin{equation}
    \int_{-\infty}^{\infty}|\Psi(x,t)|^2\,dx=1
\end{equation}

Such that if that integral does not yield one, but some other finite answer, we shall scale it with some real number $A$, like $A\Psi(x,t)$. If the result is not finite, the function is non normalizable and as such not physical. Note that the wave function should go to zero at both infinities. And $\Psi(x,t)$ should fall of faster than $1/\sqrt{|x|}$ in order to be normalizable. It is also important to note that if a wave function is normalized at any time $t$ it is also normalized for all future time.

\subsection{Momentum}

The expectation value of $x$ is given by

\begin{equation}
    \langle x \rangle = \int x\Psi^{*}\Psi\,dx
\end{equation}

Such that that of the speed $v$ is given as

\begin{equation}
    \langle v \rangle = \frac{d\langle x\rangle}{dt}=\int x\frac{\partial}{\partial t}\big(\Psi^{*}\Psi\big)\,dx=\frac{-i\hbar}{m}\int\Psi^{*}\frac{\partial\Psi}{\partial x}\,dx
\end{equation}

Such that the expectation value of momentum is given as

\begin{equation}
    \langle p \rangle = m\langle v \rangle = -i\hbar\int\Psi^{*}\frac{\partial\Psi}{\partial x}\,dx
\end{equation}

From these equations we extrapolate that there are two important operators, namely $x$, representing position, and $-i\hbar(\partial/\partial x)$ for momentum. In general any observable is a function of $x$ and $p$, $Q=Q(x, p)$. And in QM

\begin{equation}
    \langle Q \rangle = \int\Psi^{*}Q\bigg(x, -i\hbar\frac{\partial}{\partial x}\bigg)\Psi\,dx
\end{equation}

\subsection{H.U.P.}

A general property of waves is that they cannot have a definite position and a clear wavelength. In QM the wavelength of $\Psi$ is related to $p$ by the \textbf{de Broglie formula}:

\begin{equation}
    p=\frac{h}{\lambda}=\frac{2\pi h}{\lambda}
\end{equation}

So like waves, in QM we cannot know both the position and momentum in detail. It is one or the other. Quantitatively this is described by \textbf{Heisenberg's uncertainty principle}:

\begin{equation}
    \sigma_x\sigma_p\ge\frac{\hbar}{2}
\end{equation}

\section{Time-independent Schrodinger equation}

\subsection{Stationary states}

lets assume that $V=V(x)$, and let $\Psi(x,t)=\psi(x)\varphi(t)$. Now we can attempt to solve the Schrodinger equation by separation of variables.

\bigskip

\paragraph{1.} Note that

\begin{equation}
	\frac{\partial\Psi}{\partial t}=\psi\frac{d\varphi}{dt}\quad\text{and}\quad\frac{\partial^2\Psi}{\partial x^2}=\varphi\frac{d^2\psi}{dx^2}
\end{equation}

\paragraph{2.} Now we can write the Schrodinger equation as

\begin{equation}
	i\hbar\psi\frac{d\varphi}{dt}=-\frac{\hbar^2}{2m}\frac{d^2\psi}{dx^2}\varphi+V\psi\varphi
\end{equation}

\paragraph{3.} Then divide by $\psi\varphi$, and we obtain

\begin{equation}
i\hbar\frac{1}{\varphi}\frac{d\varphi}{dt}=-\frac{\hbar^2}{2m}\frac{d^2\psi}{dx^2}\frac{1}{\psi}+V
\end{equation}

Now the LHS is just a function of time and the RHS just a function of position.

\paragraph{4.} Now we let the separation constant be $E$, such that we obtain

\setlength{\leftskip}{0.5cm}

\paragraph{4.1}

\begin{equation}
	i\hbar\frac{1}{\varphi}\frac{d\varphi}{dt}=E\quad\Rightarrow\quad\varphi(t)=e^{-iEt/\hbar}
\end{equation}

\paragraph{4.2}

\begin{equation}
	-\frac{\hbar^2}{2m}\frac{d^2\psi}{dx^2}+V\psi=E\psi
\end{equation}

\setlength{\leftskip}{0cm}

This (last) equation is the time-independent Schrodinger equation; to solve it we first need to know $V(x)$.

\bigskip

One might ask why these separable solutions are so great, well

\begin{enumerate}
	\item They are stationary states: $|\Psi|^2$ and $\langle Q(x, p)\rangle$ do not depend on time, thus we can often drop $\varphi(t)$.
	\item They are states of definite total energy: In CM the Hamiltonian is defined as $$H=T+V=\frac{p^2}{2m}+V(x)$$ By applying the operator for the momentum as defined in section 1.4 we define the Hamiltonian operator: $$\hat{H}=-\frac{\hbar^2}{2m}\frac{\partial^2}{\partial x^2}+V$$ Such that the time-independent Schrodinger equation can be rewritten as $$\hat{H}\psi=E\psi$$ The expectation value $\langle H\rangle=E$, and note that $$\hat{H}^2\psi=E^2\psi$$ Such that $$\sigma_H^2=\langle H^2\rangle-\langle H\rangle^2=E^2-E^2=0$$ So a separable solution is certain to have a total energy $E$ for every measurement.
	\item The general solution is a linear combination of separable solutions: the time-independent Schrodinger equation yields an infinite collection of solutions $\{\psi_n(x)\}$, and each is associated with a specific energy $\{E_n\}$. Such that $$\Psi(x,t)=\sum_{n=1}^{\infty}c_n\psi_n(x)e^{iE_nt/\hbar}$$ Note that $|c_n|^2$ is the probability that a measurement of the energy returns $E_n$. And logically $$\sum |c_n|^2=1\quad\Rightarrow\quad\langle H\rangle=\sum |c_n|^2E_n$$
\end{enumerate}

\subsection{Infinite square well}

Lets now assume a certain potential

\[V(x) = \left\{
\begin{array}{lr}
0, & \text{if}\ 0 \le x \le a\\
\infty, & \text{otherwise}
\end{array}
\right.
\]

Thus the particle(s) is/are stuck between $x=0$ and $x=a$. 

\paragraph{Outside the well: } $\psi(x)=0$
\paragraph{Inside the well: }

\begin{align}
	V=0\quad&\Rightarrow\quad -\frac{\hbar^2}{2m}\frac{d^2\psi}{dx^2}=E\psi \\[1em]
	&\Rightarrow\quad \frac{d^2\psi}{dx^2}=-k^2\psi\ ,\ \ k^2=\frac{2mE}{\hbar^2} 
\end{align}

Note that we assume that $E\ge 0$. This is just a simple harmonic oscillator, thus the solution is given by

\begin{equation}
	\psi(x)=A\sin kx+B\cos kx
\end{equation} 

We have the boundary condition $\psi(0)=\psi(a)=0$, therefore $B=0$. Thus 

\begin{equation}
	\psi(x)=A\sin kx
\end{equation}

and since $A\not=0$ $ka=n\pi$. So

\begin{equation}
	k_n=\frac{n\pi}{a}
\end{equation}

Now the energies $E_n$ are given as

\begin{align}
	E_n=\frac{\hbar^2k_n^2}{2m}=\frac{\hbar^2\pi^2n^2}{2ma^2}
\end{align}

We arrive at this expression by noting that $E=T+V=T+0=p^2/2m$, and $p=2\pi\hbar/\lambda=k\hbar$ where $k$ is the wave number $k=2\pi/\lambda$. Such that $p_n=\hbar k_n\Rightarrow p_n^2=\hbar^2k_n^2$, which we then plug into $E_n=p_n^2/2m$.

\bigskip

Now we still need to find $A$, which we do by normalizing $\psi(x)$:

\begin{equation}
	|A|^2\int_0^a\sin^2 kx\,dx=|A|^2\frac{a}{2}=1\quad\Rightarrow\quad A=\sqrt{\frac{2}{a}}
\end{equation}

As such we now have $\psi_n(x)$

\begin{equation}
	\psi_n(x)=\sqrt{\frac{2}{a}}\sin\bigg(\frac{n\pi}{a}x\bigg)
\end{equation}

Where $\psi_1$ is the ground state and $\{\psi_n\}$, for $n>1$, are the excited states. A few properties of $\psi_n(x)$ are

\begin{enumerate}
	\item $\psi_n$ is alternately even and odd.
	\item As energy increases, each successive state has one more node (zero-crossing).
	\item $$\int\psi_m(x)^*\psi_n(x)\,dx=\delta_{mn}$$ Thus 0 for $m\not=n$ and 1 for $m=n$.
	\item They are complete, meaning that you can express any $f(x)$ as a linear combination of $\psi_n$, $$f(x)=\sum_{n=1}^{\infty}c_n\psi_n(x)\ , \ \ c_n=\int\psi_n(x)^*f(x)\,dx$$
\end{enumerate}

Thus, given the potential $V(x)$ we find $\Psi_n(x, t)$ to be

\begin{equation}
	\Psi_n(x,t)=\sqrt{\frac{2}{a}}\sin\bigg(\frac{n\pi}{a}x\bigg)e^{-iE_nt/\hbar}\ , \ \ E_n=\frac{\hbar^2\pi^2n^2}{2ma^2}
\end{equation}

And therefore the wave function $\Psi(x, t)$ is given by

\begin{equation}
\Psi(x,t)=\sum_{n=1}^{\infty}c_n\Psi_n(x,t)=\sum_{n=1}^{\infty}c_n\sqrt{\frac{2}{a}}\sin\bigg(\frac{n\pi}{a}x\bigg)e^{-iE_nt/\hbar}
\end{equation}

where

\begin{equation}
	c_n=\sqrt{\frac{2}{a}}\int_0^a\sin\bigg(\frac{n\pi}{a}x\bigg)\Psi(x,0)\,dx
\end{equation}

\subsection{The harmonic oscillator}

In CM we have a harmonic oscillator whose motion is governed by

\begin{equation}
	m\ddot{x}=-kx
\end{equation}

with a solution
\begin{equation}
	x(t)=A\sin(\omega t)+B\cos(\omega t)\ , \ \ \omega\equiv\sqrt{\frac{k}{m}}
\end{equation}

with a potential 

\begin{equation}
	V(x)=\frac{1}{2}kx^2
\end{equation}

In QM we want to solve the Schrodinger equation for the potential

\begin{equation}
	V(x)=\frac{1}{2}m\omega^2x^2
\end{equation}

Substituting it into the time-independent Schrodinger equation yields

\begin{equation}
	-\frac{\hbar^2}{2m}\frac{d^2\psi}{dx^2}+\frac{1}{2}m\omega^2x^2\psi=E\psi
\end{equation}

We shall solve it by two methods.

\subsubsection{Algebraic method}

We can rewrite it as

\begin{equation}
	\frac{1}{2m}\big[\hat{p}^2+(m\omega x)^2\big]\psi=E\psi
\end{equation}

where $\hat{p}\equiv (\hbar/i)d/dx$, the momentum operator. We'd like to factor the Hamiltonian

\begin{equation}
	\hat{H}=\frac{1}{2m}\big[\hat{p}^2+(m\omega x)^2\big]
\end{equation}

In order to do so we consider the quantity

\begin{equation}
	\hat{a}_{\pm}\equiv \frac{1}{\sqrt{2\hbar m\omega}}\big(\mp ip + m\omega x\big)
\end{equation}

Now what is $\hat{a}_{-}\hat{a}_{+}$?

\begin{equation}
	\hat{a}_{-}\hat{a}_{+}=\frac{1}{2\hbar m\omega}\big[p^2+(m\omega x)^2-im\omega(x\hat{p}-\hat{p}x)\big]
\end{equation}

Where $x\hat{p}-\hat{p}x$ is called the commutator of $x$ and $\hat{p}$, written as $[x,\hat{p}]$. In general $[\hat{A},\hat{B}]=\hat{A}\hat{B}-\hat{B}\hat{A}$. So

\begin{equation}
	\hat{a}_{-}\hat{a}_{+}=\frac{1}{2\hbar m\omega}\big[p^2+(m\omega x)^2\big]-\frac{i}{2\hbar}\big[x, \hat{p}\big]
\end{equation}

Where $\big[x, \hat{p}\big]=i\hbar$. Such that we have that

\begin{equation}
	\hat{a}_-\hat{a}_+=\frac{1}{\hbar\omega}\hat{H}+\frac{1}{2}\quad\Rightarrow\quad\hat{H}=\hbar\omega\bigg(\hat{a}_-\hat{a}_+-\frac{1}{2}\bigg)
\end{equation}

and

\begin{equation}
\hat{a}_+\hat{a}_-=\frac{1}{\hbar\omega}\hat{H}-\frac{1}{2}\quad\Rightarrow\quad\hat{H}=\hbar\omega\bigg(\hat{a}_+\hat{a}_-+\frac{1}{2}\bigg)
\end{equation}
 
Thus one sees that $[\hat{a}_-,\hat{a}_+]=1$. We can now rewrite the Schrodinger equation as

\begin{equation}
	\hbar\omega\bigg(\hat{a}_\pm \hat{a}_\mp\pm\frac{1}{2}\bigg)\psi=E\psi
\end{equation}

$\hat{a}_\pm$ is called the ladder operator, if the operator acts on the wave function it will either increase or decrease the energy level by $\hbar\omega$. $\hat{a}_-$ is the lowering operator and $\hat{a}_+$ is the raising operator. To continue we first determine $\psi_0$ for which $a_-\psi_0=0$. Such that we find

\begin{equation}
	\psi_0(x)=Ae^{-\frac{m\omega}{2m}x^2}\ , \ \ A=\bigg(\frac{m\omega}{\pi\hbar}\bigg)^{1/4}
\end{equation}

Where $A$ was found by normalizing $\psi_0(x)$. The energy of this state, that is the ground state, is $E_0=(1/2)\hbar\omega$. Then every step up the energy ladder will be with increments of $\hbar\omega$,

\begin{equation}
	E_n=\bigg(n+\frac{1}{2}\bigg)\hbar\omega
\end{equation}

and $\psi_n$ can be determined by applying the raising operator:

\begin{equation}
	\psi_n(x)=A_n(a_+)^n\,\psi_0(x)
\end{equation}

Now to get $A_n$ we note that $\hat{a}_\pm\psi_n$ is proportional to $\psi_{n\pm 1}$, thus

\begin{equation}
	\hat{a}_+\psi_n=c_n\psi_{n+1}\quad\text{and}\quad\hat{a}_-\psi_n=d_n\psi_{n-1}
\end{equation}

We find that $c_n=\sqrt{n+1}$ and $d_n=\sqrt{n}$, such that

\begin{equation}
	\hat{a}_+\psi_n=\sqrt{n+1}\psi_{n+1}\quad\text{and}\quad\hat{a}_-\psi_n=\sqrt{n}\psi_{n-1}
\end{equation}

Such that $\psi_n$ is given by

\begin{equation}
	\psi_n(x)=\frac{1}{\sqrt{n!}}(a_+)^n\,\psi_0(x)
\end{equation}

\subsection{Analytic method}

Let

\begin{equation}
	\xi\equiv\sqrt{\frac{m\omega}{\hbar}}x
\end{equation}

Such that the Schrodinger equation becomes

\begin{equation}
	\frac{d^2\psi}{d\xi^2}=\big(\xi^2-K\big)\psi
\end{equation}

If we assume that $\xi^2>>K$ then

\begin{equation}
	\frac{d^2\psi}{d\xi^2}\approx\xi^2\psi
\end{equation}

Which has a (valid) solution equal to

\begin{equation}
	\psi(\xi)=h(\xi)e^{-\xi^2/2}
\end{equation} 

Substituting this back into the time-independent Schrodinger equation yields

\begin{equation}
\frac{d^2h}{d\xi^2}-2\xi\frac{dh}{d\xi}+(K+1)h=0
\end{equation}

To solve this equation we shall try a solution of the form

\begin{equation}
	h(\xi)=\sum_{j=0}^{\infty}a_j\xi^j
\end{equation}

Such that

\begin{align}
	\frac{dh}{d\xi}&=\sum_{j=0}^{\infty}ja_j\xi^j-1\\[0.5em]
	\frac{d^2h}{d\xi^2}&=\sum_{j=0}^{\infty}(j+1)(j+2)a_{j+2}\xi^j
\end{align}

So we have that

\begin{equation}
	(j+1)(j+2)a_{j+2}-2ja_j+(K-1)a_j=0\quad\Rightarrow\quad a_{j+2}=\frac{(2j+1-K)}{(j+1)(j+2)}a_j
\end{equation}

The solution is just a combination of the even and odd terms:

\begin{equation}
	h(\xi)=h_{even}(\xi)+h_{odd}(\xi)
\end{equation}

Where 

\begin{align}
	h_{even}&=a_0+a_2\xi^2+a_4\xi^4+\dots\\[0.5em]
	h_{even}&=a_1\xi+a_3\xi^3+\dots
\end{align}

For physical solutions it is required that $K=2n+1$, such that we again obtain that

\begin{equation}
	E_n=\bigg(n+\frac{1}{2}\bigg)\hbar\omega
\end{equation}

Therefore

\begin{equation}
 a_{j+2}=\frac{-2(n-j)}{(j+1)(j+2)}a_j
\end{equation}

Such that we in the end arrive at the solution

\begin{equation}
	\psi_n(x)=\bigg(\frac{m\omega}{\pi\hbar}\bigg)^{1/4}\frac{1}{\sqrt{2^n n!}}H_n(\xi)e^{-\xi^2/2}
\end{equation}

Where $H_n$ is called the Hermite polynomial. E.g. $H_0=1$, $H_1=2\xi$, and $H_3=4\xi^2-12\xi$.

\bigskip

\paragraph{Note:} $n$ signifies the highest $j$, such that $a_{n+2}=0$, this will truncate either the odd or even series; the other one must be zero from the start. $a_1=0$ if $n$ is even and $a_0=0$ if $n$ is odd.

\subsection{The free particle}

We consider the case where $V(x)=0$, the free particle. The beginning is similar to the infinite potential well, but we try a solution of the form

\begin{equation}
	\psi(x)=Ae^{ikx}+Be^{-ikx}
\end{equation}

Note that since there are no boundary conditions; the free particle can have any positive energy. Such that

\begin{equation}
\psi(x)=Ae^{ik\big(x-\frac{\hbar k}{2m}t\big)}+Be^{-ik\big(x-\frac{\hbar k}{2m}t\big)}\quad\Rightarrow\quad \Psi_k(x,t)=Ae^{i\big(kx-\frac{\hbar k^2}{2m}t\big)}
\end{equation}

where

\begin{equation}
	k\equiv\pm\sqrt{\frac{2mE}{\hbar}}
\end{equation}

Which means that for $k>0$ the waves are traveling to the right, and for $k<0$ are traveling to the left. So note that the stationary states in this scenario are propagating waves, with wavelengths $\lambda=2\pi/|k|$, for which $p=\hbar k$. They propagate with a speed 

\begin{equation}
	v_{quantum}=\sqrt{\frac{E}{2m}}
\end{equation}

Which is twice that of its classical counterpart. Now it is important to realize that this wave function is not normalizable. As such there is no such thing as a free particle with definite energy. The full solution is

\begin{equation}
	\Psi(x,t)=\frac{1}{\sqrt{2\pi}}\int_{-\infty}^{\infty}\psi(k)e^{i\big(kx-\frac{\hbar k^2}{2m}t\big)}\,dk
\end{equation}

Note that $c_n\rightarrow (1/\sqrt{2\pi})\phi(k)\,dk$, this wave function can be normalized for appropriate $\phi(k)$. We now want to find $\phi(k)$ so as to match $\Psi(x,0)$.

\begin{equation}
	\Psi(x,0)=\frac{1}{\sqrt{2\pi}}\int_{-\infty}^{\infty}\phi(k)e^{ikx}\,dk
\end{equation}

The solution is given by the Fourier transform

\begin{equation}
	\phi(k)=\frac{1}{\sqrt{2\pi}}\int_{-\infty}^{\infty}\Psi(x,0)e^{-ikx}\,dx
\end{equation}

\subsection{The delta function potential}

Let's consider a potential of the form 

\begin{equation}
	V(x)=-\alpha\delta(x)
\end{equation}

where $\delta(x)$ is the Dirac delta function. Such that the Schrodinger equation becomes

\begin{equation}
	-\frac{\hbar^2}{2m}\frac{d^2\psi}{dx^2}-\alpha\delta(x)\psi=E\psi
\end{equation}

First we'll look at the bound states ($E<0$), for $x<0$:

\subsection{The finite square well}

\[V(x) = \left\{
\begin{array}{lr}
-V_0, &-a \le x \le a\\
0, &|x|>a
\end{array}
\right.
\]

$V_0>0$, such that we have bound states with $E<0$ and scatter states with $E>0$. For $x<-a$, 

\begin{equation}
	\frac{d^2\psi}{dx^2}=\kappa^2\psi, \ \text{where}\ \ \kappa=\frac{\sqrt{-2mE}}{\hbar}
\end{equation}

Note that $\kappa$ is positive and real. With solution $\psi(x)=Be^{\kappa x}$, the other term is left out since it will blow up in the limit for $x\rightarrow -\infty$. For $-a\le x\le a$ we have


\begin{equation}
\frac{d^2\psi}{dx^2}=-l^2\psi, \ \text{where}\ \ \kappa=\frac{\sqrt{2m(E+V_0)}}{\hbar}
\end{equation}

with solution $\psi(x)=C\sin(lx)+D\cos(lx)$. Lastly, for $x>a$ the solution is given as $\psi(x)=Fe^{\kappa x}$. Since $\psi(x)$ is continuous at the boundaries we get the equation $\kappa=l\tan(la)$. This is a formula for the allowed energies. Now let $z\equiv la$ and $z_0=(a/\hbar )\sqrt{2mV_0}$. Such that 

\begin{equation}
	\tan z = \sqrt{\bigg(\frac{z_0}{z}\bigg)^2-1}
\end{equation}

This is a transcendental equation for $z$ (thus also for the energies $E$). Now there are two cases of interest:

\begin{enumerate}
	\item Wide, deep well: If $z_0$ is very large, the intersections (in a graph of both sides of equation (75) versus $z$, these intersections are the allowed energies). These occur at $z_n\approx n\pi/2$, so $$E_n+V_0\approx\frac{n^2\pi^2\hbar^2}{2m(2a)^2}$$ for $n=1,3,5,\dots$ Note that for any finite $V_0$ there are only a finite number of bound states.
	\item Shallow, narrow well: As $z_0$ (which is a measure for the well's size) decreases, there will be fewer and fewer states. Until there is one left, there can never be zero bound states.  
\end{enumerate}

Now onto the scatter states, where $E>0$, for $x<-a$ we obtain

\begin{equation}
	\psi(x)=Ae^{ikx}+Be^{-ikx}, \ \text{where}\ \ \kappa=\frac{\sqrt{2mE}}{\hbar}
\end{equation}

For $-a\le x\le a$ nothing changes. And for $x>a$ we have $\psi(x)=Fe^{ikx}$. By application of the boundary conditions we can obtain the transmission coefficient

\begin{equation}
	T^{-1} = 1+\frac{V_0^2}{4E(E+V_0)}\sin^2\bigg(\frac{2a}{\hbar}\sqrt{2m(E+V_0)}\bigg)
\end{equation}

Note the well becomes transparent ($T=1$) when the sine is zero, so for

\begin{equation}
	\frac{2a}{\hbar}\sqrt{2m(E+V_0)}=n\pi
\end{equation}

for $n$ is any integer. Thus the energies for perfect transmission are given as

\begin{equation}
	E_n+V_0=\frac{n^2\pi^2\hbar^2}{2m(2a)^2}
\end{equation}

\section{Formalism}

\subsection{Hilbert spaces}

In QM the state of a system is determined by its wave function, observables are represented by operators. And the language in which these communicate is the mathematical language of linear algebra. Such that a column vector $|\alpha\rangle$ is given as

\begin{equation}
	|\alpha\rangle=\begin{pmatrix} a_1 \\ a_2 \\ \vdots \\ a_n\end{pmatrix}
\end{equation}

And the inner product (which in general is a complex number) is defined as

\begin{equation}
	\langle\alpha |\beta\rangle = a_1^*b_1 + a_2^*b_2 + \dots + a_n^*b_n
\end{equation}

Remember that for physical states 

\begin{equation}
	\int |\Psi|^2\,dx=1
\end{equation}

So $\Psi$ is square integrable. In general the functions $f(x)$, that are square integrable, so

\begin{equation}
	\int_a^b |f(x)|^2\,dx<\infty
\end{equation}

constitute a vector space, this vector space is called a \textbf{Hilbert space}. Thus wave functions live in \textbf{Hilbert space}. (Note that often $a$ and $b$ are $\pm\infty$.) The inner product in Hilbert space is defined as follows, for two functions $f(x)$ and $g(x)$

\begin{equation}
	\langle f | g \rangle\equiv\int_a^b f(x)^*g(x)\,dx
\end{equation}

If both functions are in Hilbert space then the inner product definitely exists. Note that (1) $\langle g | f \rangle=\langle f | g \rangle^*$, (2)

\begin{equation}
	\langle f | f \rangle\ge 0
\end{equation}

(3) a function $f(x)$ is normalized iff $\langle f | f \rangle=1$ (4) lastly a set of functions is complete if any other function in Hilbert space can be expressed as a linear combination of them:

\begin{equation}
	f(x)=\sum c_nf_n(x)
\end{equation}

and if the functions $\{f_n(x)\}$ are orthonormal then the coefficients are given by

\begin{equation}
	c_n=\langle f_n|f\rangle
\end{equation}

\subsection{Observables}

The expectation value of an observable $Q(x,p)$ can be rewritten using the inner product notation:

\begin{equation}
	\langle Q \rangle=\langle \Psi | \hat{Q}\Psi \rangle
\end{equation}

Note that $Q$ is real so $\langle Q \rangle=\langle Q \rangle^*$, but note that

\begin{equation}
	\langle \Psi | \hat{Q}\Psi \rangle=\langle \hat{Q}\Psi | \Psi \rangle
\end{equation}

Thus for any observable, given some function $f(x)$ it holds that

\begin{equation}
	\langle f | \hat{Q}f \rangle=\langle \hat{Q}f | f \rangle, \ \ \forall f(x)
\end{equation}

Such an operator is called \textbf{hermitian}. Thus observables are represented by hermitian operators. Note that the \textbf{hermitian conjugate} of an operator $\hat{Q}$ is the operator $\hat{Q}^\dagger$ such that

\begin{equation}
	\langle f | \hat{Q}g \rangle=\langle \hat{Q}^\dagger f | g \rangle, \ \ \forall f(x),g(x)
\end{equation}

So a hermitian operator has the property that $\hat{Q}=\hat{Q}^\dagger$. (In general, for matrices the hermitian conjugate is the transpose of the complex conjugate of said matrix.) A question one might ask is if there is any way to measure the same value $q$ for an observable $Q$ every time a system in a similar state as another system were measured. The answer is yes. These determinate states of $Q$ are eigenfunctions of $\hat{Q}$. Such that

\begin{equation}
	\hat{Q}\Psi=q\Psi
\end{equation}

Thus a measurement on such a state will certainly yield eigenvalue $q$. The collection of all the eigenvalues of an operator are called a spectrum, if linearly independent eigenfunctions yield the same eigenvalue then those spectra are said to be degenerate.

\end{document}

