\documentclass[a4paper]{article}

\usepackage[english]{babel}
\usepackage[utf8]{inputenc}
\usepackage{amsmath}
\usepackage{amssymb}
\usepackage{caption}
\usepackage{siunitx}
\usepackage{graphicx}
\usepackage{textcomp}
\usepackage{gensymb}
\graphicspath{{./}}
\usepackage[colorinlistoftodos]{todonotes}
\usepackage[section]{placeins}
\setlength{\parindent}{0pt}
\usepackage{url}
\usepackage{bm}
\usepackage{mathtools}
\usepackage{enumerate}

\title{Lecture Notes Thermal Physics}
\author{Mick Veldhuis}
\date{\today}

\begin{document}
\maketitle

\tableofcontents

\section{Introduction: a short overview of important notions}

\subsection*{The mole}

A \textbf{mole} of a certain atom is equivalent to an \textbf{Avogadro number} $N_a$ of atoms. Where $N_A=6.022\cdot 10^23$.

\subsection*{The ideal gas law}

Suppose we have $N$ particles in a gas, then we can relate the pressure, volume, temperature, and amount of particles as follows:

\begin{equation}
	pV=Nk_BT
\end{equation}

where $k_B$ is Boltzmann's constant. In order to apply this formula one assumes that (i) there are no intermolecular forces and (ii) that the particles are point-like and have zero size. 

\subsection*{Combinatorics}

Suppose we have $n$ atoms, and $r$ of those atoms are for instance in an excited state. If we want to calculate the number of possible configurations $\Omega$ we apply the following formula,

\begin{equation}
	\Omega=\frac{n!}{(n-r)!r!}\equiv {}^nC_r
\end{equation}

Since factorials grow incredibly quick, we shall often use $\ln\Omega$ instead of $\Omega$. 

\begin{equation}
	\ln\Omega=\ln(n!)-\ln([n-r]!)-\ln(r!)
\end{equation}

We also have \textbf{Sterling's formula}:

\begin{equation}
	\ln(n!)\approx n\ln n-n
\end{equation}

\section{Heat}


\end{document}

