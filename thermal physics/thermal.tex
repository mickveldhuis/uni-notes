\documentclass[a4paper]{article}

\usepackage[english]{babel}
\usepackage[utf8]{inputenc}
\usepackage{amsmath}
\usepackage{amssymb}
\usepackage{caption}
\usepackage{siunitx}
\usepackage{graphicx}
\usepackage{textcomp}
\usepackage{gensymb}
\graphicspath{{./}}
\usepackage[colorinlistoftodos]{todonotes}
\usepackage[section]{placeins}
\setlength{\parindent}{0pt}
\usepackage{url}
\usepackage{bm}
\usepackage{mathtools}
\usepackage{enumerate}

\title{Lecture Notes Thermal Physics}
\author{Mick Veldhuis}
\date{\today}

\begin{document}
\maketitle

\tableofcontents

\section{Introduction: a short overview of important notions}

\subsection*{The mole}

A \textbf{mole} of a certain atom is equivalent to an \textbf{Avogadro number} $N_a$ of atoms. Where $N_A=6.022\cdot 10^{23}$.

\subsection*{The ideal gas law}

Suppose we have $N$ particles in a gas, then we can relate the pressure, volume, temperature, and amount of particles as follows:

\begin{equation}
	pV=Nk_BT
\end{equation}

where $k_B$ is Boltzmann's constant. In order to apply this formula one assumes that (i) there are no intermolecular forces and (ii) that the particles are point-like and have zero size. 

\subsection*{Combinatorics}

Suppose we have $n$ atoms, and $r$ of those atoms are for instance in an excited state. If we want to calculate the number of possible configurations $\Omega$ we apply the following formula,

\begin{equation}
	\Omega=\frac{n!}{(n-r)!r!}\equiv {}^nC_r
\end{equation}

Since factorials grow incredibly quick, we shall often use $\ln\Omega$ instead of $\Omega$. 

\begin{equation}
	\ln\Omega=\ln(n!)-\ln([n-r]!)-\ln(r!)
\end{equation}

We also have \textbf{Sterling's formula}:

\begin{equation}
	\ln(n!)\approx n\ln n-n
\end{equation}

\section{Heat}

Heat ($Q$, measured in Joules) is thermal energy in transit. Note that in contrast to for instance fuel in a car, one cannot say that something has a certain quantity of heat. One can only add heat to something, like adding fuel to a car.

\subsection{Heat capacity}

We can ask the question, how much heat needs to be supplied to an object tot raise its temperature by a small amount d$T$. The answer is $\text{d}Q=C\text{d}T$, such that we define the \textbf{heat capacity} $C$ of an object using

\begin{equation}
	C=\frac{dQ}{dT}
\end{equation}

There is also $c$, the \textbf{specific heat capacity} in $\si{\joule\per\kelvin\per\kilo\gram}$. And the \textbf{molar heat capacity}, the heat capacity of one mole of a substance. 

\bigskip

If we were to add heat to a system, we'd have to constrain, either the volume of the gas or the pressure of the gas to be fixed. We should therefore change the definition of heat capacity, we need new quantities: $C_V$, the heat capacity at constant volume, and $C_p$, the heat capacity at constant pressure.

\begin{align}
	C_V&=\bigg(\frac{\partial Q}{\partial T}\bigg)_V\\[1em]
	C_p&=\bigg(\frac{\partial Q}{\partial T}\bigg)_p
\end{align}

In practice $C_p>C_V$, due to more heat being added whilst heating at constant pressure. While heating at constant volume will expend additional energy on doing work on the atmosphere as the gas expands. 

\section{Probability}


\subsection{Discrete probability distributions}

Let $x$ be a discrete random variable which takes values $x_i$ with probability $P_i$. The sum of $P_i$'s must always be $\sum P_i=1$. And we define the \textbf{mean} of $x$ to be

\begin{equation}
	\langle x \rangle = \sum_i x_iP_i
\end{equation}

and in general

\begin{equation}
\langle f(x) \rangle = \sum_i f(x_i)P_i
\end{equation}

\subsection{Continuous probability distributions}

Let $x$ be a continuous random variable which has a probability $P(x)\,dx$ of having a value between $x$ and $x+dx$. We require that

\begin{equation}
	\int P(x)\,dx=1
\end{equation}

With the mean defined as

\begin{equation}
\langle x \rangle = \int xP(x)\,dx
\end{equation}

and in general

\begin{equation}
\langle f(x) \rangle = \int f(x)P(x)\,dx
\end{equation}

\subsection{Linear transformation}

If one want to relate say a random variable $x$ to the random variable $y$ by e.g. $y=ax+b$ then

\begin{equation}
\langle y \rangle = \langle ax+b \rangle= a\langle x \rangle+b
\end{equation}

where $a$ and $b$ are just constants.

\subsection{Variance}

One measurement of spread in a distribution is the variance (the square root of it is the standard deviation), defined as

\begin{equation}
	\sigma_x^2=\langle x^2 \rangle - \langle x \rangle^2
\end{equation}

\subsection{Independent variables}

If $u$ and $v$ are independent random variables the probability that they are in there respective range is given by

\begin{equation}
	P_u(u)\,du\,P_v(v)\,dv
\end{equation}

and by taking the double integral of that product we obtain

\begin{equation}
	\langle uv \rangle = \langle u \rangle \langle v \rangle
\end{equation}


\subsection{Binomial distribution}

It is a probability distribution based on \textbf{Bernoulli trials}, which are experiments with two outcomes, success and failure. Where the success happens with a probability $p$, and failure with probability $1-p$. The \textbf{binomial distribution} is the discrete probability distribution of getting $k$ successes from $n$ independent Bernoulli trials. It is defined as

\begin{equation}
	P(n, k) = p^k(1-p)^{n-k}\big({}^nC_k\big)\quad\text{where}\quad {}^nC_k=\frac{n!}{(n-k)!k!}
\end{equation}

and the mean of the distribution is $\langle k \rangle=np$ and the variance is $\sigma_k^2=np(1-p)$.

\section{Temperature and the Boltzmann factor}

\subsection{Thermal equilibrium}

Suppose two bodies are in thermal contact with one another, heat will flow from the hotter to the colder body. At a certain moment no net flow of heat will happen, the energy and temperature will no longer change with time. This is called the \textbf{Thermal equilibrium}, the process of reaching it is called \textbf{thermalization}.

\subsection{The zeroth law of thermodynamics}

It says that two systems, each, separately in thermal equilibrium with a third, are in equilibrium with each other. 

\subsection{On thermometers}

A working thermometer should have a heat capacity that is less than that of the measured object. Some common methods of measuring temperature include (i) the fact that liquids expand, (ii) that the resistance of certain materials depend on temperature, or (iii) letting liquid coexist with the vapour and measuring the pressure of the vapour.

\subsection{Microstates and macrostates}

Let's consider the example of a hundred coin flips. Here the \textbf{microstates} would be either heads or tails (note microstates are all equally likely). Macrostates would be the total outcome, so how many heads or tails (these are not equally likely).

\subsection{The statistical definition of temperature}

Let's imagine two systems in thermal contact, otherwise they are isolated. The total energy is $E=E_1+E_2=\text{constant}$. The first system can be in any one of $\Omega_1(E_1)$ microstates and the second system can be in any one of $\Omega_2(E_2)$. The total system can then be in any of $\Omega_1(E_1)\Omega_2(E_2)$ microstates.

\bigskip

If we assume that the system is in thermal equilibrium, the energies $E_1$ and $E_2$ are fixed. To continue we'll try to maximize $\Omega_1(E_1)\Omega_2(E_2)$.

\begin{align}
	\frac{d}{dE_1}\bigg\{\Omega_1(E_1)\Omega_2(E_2)\bigg\}=\Omega_2(E_2)\frac{d\Omega_1(E_1)}{dE_1}+\Omega_1(E_1)\frac{d\Omega_2(E_2)}{dE_2}\frac{dE_2}{dE_1}=0
\end{align}

Note that $dE_1=-dE_2$ therefore $dE_1/dE_2=-1$, since $E=E_1+E_2$. We can use that to rewrite the previous expression to

\begin{equation}
	\frac{1}{\Omega_1}\frac{d\Omega_1}{dE_1}=\frac{1}{\Omega_2}\frac{d\Omega_2}{dE_2}\quad\Rightarrow\quad\frac{d\ln\Omega_1}{dE_1}=\frac{d\ln\Omega_2}{dE_2}
\end{equation}

Let $d\ln\Omega/dE$ be identified with a temperature $T$ such that $T_1=T_2$. We can then define temperature by

\begin{equation}
	\frac{1}{k_BT}=\frac{d\ln\Omega}{dE}\quad\Rightarrow\quad\frac{1}{T}=\frac{dS}{dE}
\end{equation}

Where $S$ is called entropy. Often we will use 

\begin{equation}
	\beta\equiv\frac{1}{k_BT}
\end{equation}

since it is a frequently used quantity.

\subsection{Ensembles}

\textbf{Ensembles} are idealizations consisting of a large number of virtual copies of a system, considered all at once, each which represent a possible state of the real system. The main ensembles are

\begin{enumerate}
	\item Micro canonical ensemble: an ensemble of systems that each have the same fixed energy.
	\item Canonical ensemble: an ensemble of systems, each that exchange its energy with a large reservoir of heat. This fixes the temperature.
	\item Grand canonical ensemble: aside from the canonical ensemble also particles can be exchanged with the reservoir. This fixes the temperature and the chemical potential.
\end{enumerate}

\subsection{The canonical ensemble}

To examine the canonical ensemble we shall imagine a reservoir with energy $E-\epsilon$, where $\epsilon$ is the energy of the system and $E$ the total energy. (We shall assume that for each allowed energy there is one microstate.) The probability $P(\epsilon)$ that the system has energy $\epsilon$ is proportional to $\Omega(E-\epsilon)$ times the number microstates, which is one.


\begin{align}
	P(\epsilon)&\propto \Omega(E-\epsilon)=\Omega(E) e^{-\epsilon/k_BT}\\[1em]
    P(\epsilon)&\propto e^{-\epsilon/k_BT}
\end{align}

Which is called the \textbf{Boltzmann distribution} and $e^{-\epsilon/k_BT}$ is called the \textbf{Boltzmann factor}. If a system is in contact with a reservoir and has a microstate $r$, with energy $E_r$, then

\begin{equation}
	P(\text{microstate}\ r)=\frac{e^{-E_r/k_BT}}{\sum_i e^{-E_i/k_BT}}=\frac{e^{-E_r/k_BT}}{Z}
\end{equation}

where $Z$ is called the \textbf{partition function}.

\section{The Maxwell-Boltzmann distribution}

\subsection{Multilevel systems and the Boltzmann distribution}

Suppose you have a system with multiple energy levels, then the higher the state the smaller the occupancy and the higher the temperature, the more probable it is that higher states are occupied. This statement is summarized by

\begin{equation}
	\frac{N_i}{N_j}=e^{-\frac{E_i-E_j}{k_BT}}
\end{equation}

\subsection{The velocity distribution}

The velocity distribution is defined as the fraction of molecules with velocities between $\bm{v}$ and $\bm{v}+d\bm{v}$. Assuming we are only looking in one direction, say $x$, then the velocity distribution is given as

\begin{equation}
	g(v_x)\propto e^{-mv_x^2/2k_BT}\quad\Rightarrow\quad g(v_x)=\sqrt{\frac{m}{2\pi k_BT}}\,e^{-mv_x^2/2k_BT}
\end{equation}

The rightmost expression is found by normalization. For this distribution we have that

\begin{align}
	\langle v_x\rangle&=0\\[.5em]
	\langle |v_x|\rangle&=\sqrt{\frac{2k_BT}{\pi m}}\\[.5em]
	\langle v_x^2\rangle&=\frac{k_BT}{m}\\
\end{align}

Now to generalize, and go back to the original statement, between $\bm{v}$ and $\bm{v}+d\bm{v}$

\begin{equation}
	g(v_x)\,dv_x\,g(v_y)\,dv_y\,g(v_z)\,dv_z\propto e^{-mv^2/2k_BT}\,dv_x\,dv_y\,dv_z
\end{equation}

\subsection{The speed distribution}

Now we are looking at the molecules between $v$ and $v+dv$. Which in velocity space is a spherical shell of radius $v$, with thickness $dv$. The volume of that part of the shell is given as $4\pi v^2\,dv$. Such that the speed distribution is given as

\begin{equation}
	f(v)\,dv\propto v^2\,dv\,e^{-mv^2/2k_BT}
\end{equation}

Note that the $4\pi$ is absorbed into the proportionality. Now to normalize it, integrate from $0$ to $\infty$ (since $v>0$):

\begin{equation}
	\int_0^\infty f(v)\,dv=1\quad\Rightarrow\quad f(v)\,dv=\frac{4}{\sqrt{\pi}}\bigg(\frac{m}{2k_BT}\bigg)^{3/2}v^2\,e^{-mv^2/2k_BT}\,dv
\end{equation}

This expression is called the Maxwell-Boltzmann distribution. For which


\begin{align}
\langle v\rangle&=\sqrt{\frac{8k_BT}{\pi m}}\\[.5em]
\langle v^2\rangle&=\frac{3k_BT}{m}\\[.5em]
v_{rms}&=\sqrt{\langle v^2\rangle}\propto m^{-1/2}
\end{align}

From this we can derive an expression for the mean kinetic energy of a gas molecule:

\begin{equation}
	\langle T\rangle =\frac{1}{2}m\langle v^2\rangle=\frac{3}{2}k_BT
\end{equation}

We can also derive the value of $v$ such that $f(v)$ is at a maximum, this is simply

\begin{equation}
	\frac{df}{dv}=0\quad\Rightarrow\quad v_{max}=\sqrt{\frac{2k_BT}{m}}
\end{equation}

Such that $v_{max} < \langle v\rangle < v_{rms}$.

\section{Pressure}

Pressure is on of the most fundamental variables in the study of gasses and is defined as the force over the area on which it is acting. The pressure $p$ in a volume $V$ depends on the temperature $T$ via an equation of state $p=f(T,V,N)$, one such example is the ideal gas law.

\subsubsection{Molecular distributions}

If we consider a sphere in which molecular are moving in different directions, we define the \textbf{solid angle} $\Omega$ (measured in \textbf{steradians}; analogous to the angle $\theta$ in 2D), as

\begin{equation}
	\Omega=\frac{A}{r^2}
\end{equation}

A full sphere subtends a solid angle of $\Omega=4\pi$. Such that the fraction of molecules whose trajectories lie in an elemental solid angle $d\Omega$ is

\begin{equation}
	\frac{d\Omega}{4\pi}
\end{equation}

If we pick a specific direction then $d\Omega$ corresponds to molecules traveling at angles between $\theta$ and $\theta + d\theta$. In the end the number of molecules per unit volume given by

\begin{equation}
	nf(v)dv\,\frac{1}{2}\sin\theta\,d\theta
\end{equation}

have speeds between $v$ and $v+dv$ and are traveling at angles between $\theta$ and $\theta + d\theta$. If we consider a molecules hitting a wall with an angle $\theta$ to the normal then the number of molecules hitting it per unit area and unit time is

\begin{equation}
	v\cos\theta\,nf(v)dv\,\frac{1}{2}\sin\theta\,d\theta
\end{equation}

\subsubsection{The ideal gas law}

We can now calculate the pressure of a gas on its container. First note that each molecule hitting the walls of it have a momentum change of $2mv\cos\theta$. Such that if we multiply it with the result from equation (42) and integrate over all $v$ and over the angles $0\rightarrow\pi/2$ we get the pressure $p$:

\begin{equation}
	p=\frac{1}{3}nm\langle v^2\rangle\ , \ \ n=\frac{N}{V}
\end{equation}

Which can be rewritten as

\begin{equation}
	pV=Nk_BT\quad\Rightarrow\quad p=nk_BT
\end{equation}

The familiar ideal gas law.

\subsubsection{Dalton's law}

If you have a mixture of gases in thermal equilibrium then the total pressure is the sum of partial pressures $p_i=n_ik_BT$

\begin{equation}
	p=\bigg(\sum_{i}n_i\bigg)k_BT=\sum_{i}p_i
\end{equation}

\subsection{Effusion}

\textbf{Effusion} is the process by which gas escapes from a small hole, the rate of a gas doing so is given to be proportional to $1/\sqrt{m}$ (also called \textbf{Graham's law of effusion}).

\subsubsection{Flux}

Let's determine the molecular flux $\Phi$, the number of molecules striking a unit area per second. It is given as

\begin{equation}
	\Phi=\frac{1}{4}n\langle v\rangle=\frac{p}{\sqrt{2\pi mk_BT}}
\end{equation}

This expression shows us that indeed, the effusion rate is as given by Graham's law.

\subsubsection{Effusion rate}

The effusion rate (given a hole of area $A$) is given as $\Phi A$. Effusion preferentially selects faster molecules, therefore the speed distribution isn't Maxwellian. This is due to faster molecules having a greater probability of reaching the hole. Now

\begin{equation}
	f(v)\propto v^3 e^{-mv^2/2k_BT}
\end{equation}

Therefore the molecules in an effusing gas have a higher energy. Note for effusion the diameter of the hole has to be much less than the mean free path $\lambda$.


\end{document}

