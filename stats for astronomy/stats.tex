\documentclass[a4paper]{article}

\usepackage[english]{babel}
\usepackage[utf8]{inputenc}
\usepackage{amsmath}
\usepackage{amssymb}
\usepackage{caption}
\usepackage{siunitx}
\usepackage{graphicx}
\usepackage{textcomp}
\usepackage{gensymb}
\graphicspath{{./}}
\usepackage[colorinlistoftodos]{todonotes}
\usepackage[section]{placeins}
\setlength{\parindent}{0pt}
\usepackage{url}
\usepackage{bm}
\usepackage{mathtools}
\usepackage{enumerate}

\title{Lecture Notes Statistic for Astronomy}
\author{Mick Veldhuis}
\date{\today}

\begin{document}
\maketitle

\tableofcontents

\section{Statistics basics}

\subsection{Cox's rules}

The first rule states that the probability of something happening, let's call the event $X$, plus that of $X$ not happening is one:

\begin{equation}
	P(X)+P(\bar{X})=1
\end{equation}

The second rule (the product rule) states that the probability of events $X$ and $Y$ happening is given by:

\begin{equation}
	P(X,Y)=P(X|Y)\cdot P(Y)
\end{equation}

Where $P(X|Y)$ is the probability of $X$ given $Y$ happening.

\subsection{Bayes' theorem}

Bayes' theorem gives us the probability of the hypothesis $H$ being true given the data $D$.

\begin{equation}
	P(H|D)=\frac{P(D|H)\cdot P(H)}{P(D)}
\end{equation}

Where $P(H|D)$ is called the posterior, $P(D|H)$ the likelihood, $P(H)$ the prior, and $P(D)$ the evidence. 

\subsection{Marginalization}

Suppose you have a range of hypotheses $\{H_i\}$, where $i=0,1,2,3,\dots$. Then

\begin{equation}
	\sum_{i=0}^{N-1} P(H_i)=1
\end{equation}

Then suppose we have some nuisance parameter $X$, then 

\begin{align}
	\sum P(H_i,X) &= \sum P(H_i|X)\cdot P(X)\\[1em]
	&= P(X)\sum P(H_i|X)\\[1em]
	&= P(X)
\end{align}

Which could also be written as

\begin{align}
	P(X)=\int_{-\infty}^{\infty} P(X,Y)\,dY
\end{align}



\end{document}

